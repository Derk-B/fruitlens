\section{How to use this?}

You should have stack installed (see \url{https://haskellstack.org/}) and open a terminal in the same folder.
To make it easy to use the model we have created a web interface. It is recommended to have python and the http library installed to serve the web interface.

\begin{itemize}
  \item Download all the files from the fruits-360\_100x100 directory from\\ https://www.kaggle.com/datasets/moltean/fruits. Put these files in the root of the project.
  \item You can now go two ways from here: train the model or use the model to predict the fruit in a new image.
  \item 1. Run the training algorithm
  \begin{itemize}
      \item Run \verb|stack ghci| 
      \item Load the Convert.ihs file. Run \verb|:l lib/Convert.ihs|. 
      \item Call the convert function. Run \verb|convert|.
      \item After some time (can take half an hour depending on your hardware), a trained\_model.bin file should be saved in the root of your project. That binary contains the model values that you can use to predict fruits.
  \end{itemize}

  \item 2. Run the API and use that as an interface to detect fruits.
  \begin{itemize}
    \item Run \verb |stack ghci|
    \item Load the API.ihs file. Run \verb|:l lib/API.ihs|.
    \item Start the server with: \verb|startServer 8080|
    \item Now you should be able to send a base64 encoded value of a JPG to localhost:8080/api/fruitlens. However it is easier to use the web interface that we made.
    \item Start teh web interface by going into the fruitcam-web directory. Run \verb|cd fruitcam-web|. The easiest way to run the website is by using python. Run \verb|python3 -m http.server 3000|.
    \item Visit localhost:3000 in your browser and upload a jpg file.
  \end{itemize}
\end{itemize}
